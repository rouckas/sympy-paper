%% Submissions for peer-review must enable line-numbering 
%% using the lineno option in the \documentclass command.
%%
%% Preprints and camera-ready submissions do not need 
%% line numbers, and should have this option removed.
%%
%% Please note that the line numbering option requires
%% version 1.1 or newer of the wlpeerj.cls file.

\documentclass[fleqn,10pt,lineno]{wlpeerj} % for journal submissions
% \documentclass[fleqn,10pt]{wlpeerj} % for preprint submissions

\usepackage{lmodern}
\usepackage[T1]{fontenc}
\usepackage[utf8]{inputenc}
\usepackage[scaled=0.8]{DejaVuSansMono}

\usepackage{hyperref}
\usepackage{graphicx}
\usepackage[all]{xy}
\usepackage{amsmath}
\usepackage{caption}
\graphicspath{ {images/} }

% Makes quote characters in monospace font not be curly
\usepackage{upquote}

\usepackage{amsmath}
\usepackage{url}
\usepackage{hyperref}

% this is required for all the \url{} commands in the bib file
%\usepackage{hyperref}

% for nice units
\usepackage{siunitx}

% for images: png, pdf, etc
\usepackage{graphicx}

% for nice table formatting, i.e., /toprule, /midrule, etc
\usepackage{booktabs}

% to allow for \verb++ declarations in captions.
\usepackage{cprotect}

% to allow usage of \mathbb symbols
\usepackage{amssymb}

\usepackage{longtable}

\usepackage{listings}

\newcommand\email[1]{\href{mailto:#1}{#1}}
\title{SymPy: Symbolic Computing in Python}

\input{authors}

\keywords{Keyword1, Keyword2, Keyword3}

\begin{abstract}
  SymPy is an open source computer algebra system written in pure Python. It
  is built with a focus on extensibility and ease of use, through both
  interactive and programmatic applications. These characteristics have led
  SymPy to become a popular symbolic library for the scientific Python ecosystem.
  This paper presents the
  architecture of SymPy, a description of its features, and a discussion of
  select domain specific submodules.
\end{abstract}

\begin{document}

\flushbottom
\maketitle
\thispagestyle{empty}

\section{Introduction}

%% What sympy is, where to download etc.
%%
%% List other major CASs.
%%
%% Why SymPy.

SymPy is a full featured computer algebra system (CAS) written in the Python
programming language.
% cite Python?
It is free and open source software, being licensed under the 3-clause BSD
license.
% cite BSD?
The SymPy project was started by Ond\v{r}ej \v{C}ert\'{\i}k in 2005, and it has
since grown to over 500 contributors. Currently, SymPy is
developed on GitHub using a bazaar community
model~\cite{raymond1999cathedral}. The accessibility of the codebase and the
open community model allows SymPy to rapidly respond to the needs of the
community of users and developers.
% citation?

Python is a dynamically typed programming language that has a focus on
ease of use and readability. Due in part to this focus, it has become a popular
language for scientific
computing and data science, with a broad ecosystem of
libraries~\cite{oliphant2007python}. SymPy is itself used by many libraries
and tools to support research within a variety  domains, such as
Sage~\cite{SAGE} (pure mathematics),
yt~\cite{2011ApJS..192....9T} (astronomy and astrophysics),
PyDy~\cite{gede2013constrained} (multibody dynamics), and
SfePy~\cite{cimrman2014sfepy} (finite elements).

Unlike many CASs, SymPy does not invent its own programming language. Python
itself is used both for the internal implementation and the end user
interaction.  The exclusive usage of a single programming language makes it easier
for people already familiar with that language to use or develop SymPy.
Simultaneously, it enables developers to focus on mathematics, rather than
language design.

SymPy is designed with a strong focus on usability as a library.
Extensibility is important in its application program interface
(API) design, and thus SymPy makes no attempt to extend
the Python language itself. The goal is for users of SymPy to be able to
include SymPy alongside other Python libraries in their workflow, whether that
is in an interactive environment or programmatic use as part of a larger system.

As a library, SymPy does not have a built-in graphical user
interface (GUI). However, SymPy exposes a rich interactive display system,
including registering printers with Jupyter~\cite{perez2007ipython} frontends,
including the Notebook and Qt Console, which will render SymPy
expressions using MathJax~\cite{cervone2012mathjax} or \LaTeX{}.

The remainder of this paper discusses key components of the SymPy software.
Section~\ref{sec:architecture} discusses the architecture of SymPy.
Section~\ref{sec:features} enumerates the features of SymPy and takes a closer
look at some of the important ones. Following that, section~\ref{sec:numerics}
looks at the numerical features of SymPy and its dependency library, mpmath.
Section~\ref{sec:domain_specific} looks at the domain specific physics
submodules for performing symbolic and numerical calculations in classical mechanics
and quantum mechanics. Finally, section~\ref{sec:conclusion} concludes the paper
and discusses future work.


\section{Architecture}
\label{sec:architecture}

%TODO put some leading text here.

\subsection{Basic Usage}

% symbols, various ways to declare them

Because SymPy is built on Python, it requires that all variable names be
defined prior to use. The following statement
imports all SymPy functions into the global Python namespace.
From here on, all examples in this paper assume that this statement has been
run.

\begin{verbatim}
>>> from sympy import *
\end{verbatim}

Symbolic variables, called symbols, must be defined and assigned to
Python variables before they can be used. This is typically done through the
\texttt{symbols} function, which may create multiple symbols in a single call. For
instance,
\begin{verbatim}
>>> x, y, z = symbols('x y z')
\end{verbatim}
creates three symbols representing variables named $x$, $y$, and $z$. In this
particular instance, these symbols are all assigned to Python variables of the
same name. However, the user is free to assign them to different
Python variables, while representing the same symbol, such as
\texttt{a, b, c = symbols(\textquotesingle{}x y z\textquotesingle{})}.
In order to minimize potential confusion, though, all examples in this paper will
assume that
the symbols \verb|x|, \verb|y|, and \verb|z| have been assigned to Python variables
identical to their symbolic names.

Expressions are created from symbols using Python mathematical syntax. Note
that in Python, exponentiation is represented by \verb|**|. For instance, the
following Python code creates the expression $(x^2 - 2x + 3)/y$.

\begin{verbatim}
>>> (x**2 - 2*x + 3)/y
(x**2 - 2*x + 3)/y
\end{verbatim}

Importantly, SymPy expressions are immutable. This simplifies the design of
SymPy by allowing expression interning. It also enables expressions to be
hashed and stored in Python dictionaries, thereby permitting caching and
other features.

%% I volunteer to write this section. --Aaron
%%
%% Representing symbolic expressions using Python objects

\subsection{The Core}

A computer algebra system (CAS) represents mathematical expressions as data
structures.  For example the mathematical expression $x + y$ is represented as
a tree with three nodes, $+$, $x$, and $y$, where $x$ and $y$ are ordered
children of $+$.  As users manipulate
mathematical expressions with traditional mathematical syntax, the CAS
manipulates the underlying data structures.  Automated optimizations and
computations such as integration, simplification, etc.\ are all functions that
consume and produce expression trees.

In SymPy every symbolic expression is an instance of a Python \texttt{Basic}
class, a superclass of all SymPy types providing common methods to all SymPy
tree-elements, such as traversals.  The children of a node in the
tree are held in the \texttt{args} attribute.  A terminal or leaf node in the
expression tree has empty \texttt{args}.

For example, consider the expression $xy + 2$:
\begin{verbatim}
>>> expr = x*y + 2
\end{verbatim}
By order of operations, the parent of the expression tree for \texttt{expr} is
an addition, so it is of type \texttt{Add}. The child nodes of \texttt{expr} are
\texttt{2} and \texttt{x*y}.
\begin{verbatim}
>>> type(expr)
<class 'sympy.core.add.Add'>
>>> expr.args
(2, x*y)
\end{verbatim}

Traversing further into the expression tree grants the full expression. For
example, the first child node, given by \texttt{expr.args[0]}, is
\texttt{2}. Its class is \texttt{Integer}, and it has empty an \texttt{args}
tuple, indicating that it is a leaf node.
\begin{verbatim}
>>> expr.args[0]
2
>>> type(expr.args[0])
<class 'sympy.core.numbers.Integer'>
>>> expr.args[0].args
()
\end{verbatim}

A useful way to view an expression tree is with the \texttt{srepr} function.
This returns a string representation of an expression as valid Python code
with all the nested class constructor calls to create the given expression.
\begin{verbatim}
>>> srepr(expr)
"Add(Mul(Symbol('x'), Symbol('y')), Integer(2))"
\end{verbatim}

Every SymPy expression satisfies a key identity invariant:
\begin{verbatim}
expr.func(*expr.args) == expr
\end{verbatim}
This means that expressions are
rebuildable from their \texttt{args}.\footnote{\texttt{expr.func} is used
instead of \texttt{type(expr)} to allow the function of an expression to be
distinct from its actual Python class. In most cases the two are the same.}
We note that in SymPy, the \texttt{==} operator represents exact
structural equality, not mathematical equality. This allows one to test if
any two expressions are equal to one another as expression trees.

Python allows classes to override mathematical operators. The Python
interpreter translates the above \texttt{x*y + 2} to, roughly,
\verb|(x.__mul__(y)).__add__(2)|. Both \texttt{x} and \texttt{y}, returned
from the \texttt{symbols} function, are \texttt{Symbol} instances. The
\texttt{2} in the expression is processed by Python as a literal, and is
stored as Python's builtin \texttt{int} type. When \texttt{2} is passed to the
\verb|__add__| method of \texttt{Symbol}, it is converted to the SymPy type
\verb|Integer(2)| before being stored in the resulting expression tree. In
this way, SymPy expressions can be built in the natural way using Python
operators and numeric literals.

%%
%% Assumptions
\subsection{Logical Inference and Assumptions}

SymPy performs logical inference through its assumptions system. The
assumptions system allows users to specify that symbols have certain common
mathematical properties, such as being positive, imaginary, or integral. SymPy
is careful to never perform simplifications on an expression unless the
assumptions allow them. For instance, the identity $\sqrt{t^2} = t$ holds if
$t$ is nonnegative ($t\ge 0$). If $t$ is real, the identity $\sqrt{t^2}=|t|$
holds. However, for general complex $t$, no such identity holds.

By default, SymPy performs all calculations assuming that symbols are
complex valued. This assumption makes it easier to treat mathematical problems
in full generality.
\begin{verbatim}
>>> t = Symbol('t')
>>> sqrt(t**2)
sqrt(t**2)
\end{verbatim}

By assuming the most general case, that symbols are complex by default, SymPy
avoids performing mathematically invalid operations. However, in many cases
users will wish to simplify expressions containing terms like $\sqrt{t^2}$.

Assumptions are set on \texttt{Symbol} objects when they are created. For
instance \verb|Symbol('t', positive=True)| will create a symbol named
\texttt{t} that is assumed to be positive.
\begin{verbatim}
>>> t = Symbol('t', positive=True)
>>> sqrt(t**2)
t
\end{verbatim}

Some of the common assumptions that SymPy allows are \texttt{positive},
\texttt{negative}, \texttt{real}, \texttt{nonpositive}, \texttt{nonnegative},
\texttt{real}, \texttt{integer}, and \texttt{commutative}.\footnote{If $A$ and
$B$ are Symbols created with \texttt{commutative=False} then SymPy will keep
$A\cdot B$ and $B\cdot A$ distinct.} Assumptions on any object can be checked with the
\verb|is_|\texttt{\textit{assumption}} attributes, like \verb|t.is_positive|.

Assumptions are only needed to restrict a domain so that certain
simplifications can be performed. It is not required to make the domain match
the input of a function. For instance, one can create the object
$\sum_{n=0}^m f(n)$ as \verb|Sum(f(n), (n, 0, m))| without setting
\texttt{integer=True} when creating the Symbol object \texttt{n}.

The assumptions system additionally has deductive capabilities. The
assumptions use a three-valued logic using the Python builtin objects
\texttt{True}, \texttt{False}, and \texttt{None}. \texttt{None} represents the
``unknown'' case. This could mean that the given assumption could be either
true or false under the given information, for instance,
\verb|Symbol('x', real=True).is_positive| will give \texttt{None} because a real
symbol might be positive or it might not. It could also mean not enough is
implemented to compute the given fact. For instance,
\verb|(pi + E).is_irrational| gives \texttt{None}, because SymPy does not know
how to determine if $\pi + e$ is rational or irrational, indeed, it is an open
problem in mathematics.
% TODO: ref?


Basic implications between the facts are used to deduce assumptions. For
instance, the assumptions system knows that being an integer implies being
rational, so \verb|Symbol('x', integer=True).is_rational| returns
\texttt{True}. Furthermore, expressions compute the assumptions on themselves
based on the assumptions of their arguments. For instance, if \texttt{x} and
\texttt{y} are both created with \texttt{positive=True}, then
\verb|(x + y).is_positive| will be \texttt{True}.

%% TODO: axe this? Why mention it if we won't discuss it. Maybe move to
%% future work. - scopatz
SymPy also has an experimental assumptions system where facts are stored
separately from objects, and deductions are made with a SAT solver. We will not
discuss this system here.

%% TODO: describe how assumptions are implemented

%%
%% Extensibility
\subsection{Extensibility}

While the core of SymPy is relatively small, it has been extended to a wide variety
of domains by a broad range of contributors.  This is due in part because the
same language, Python, is used both for the internal implementation and the
external usage by users.  All of the extensibility capabilities available to
users are also utilized by SymPy itself. This eases the transition pathway from
SymPy user to SymPy developer.

The typical way to create a custom SymPy object is to subclass an existing
SymPy class, generally one of \texttt{Basic}, \texttt{Expr}, or
\texttt{Function}. All SymPy classes used for expression trees\footnote{Some
  internal classes, such as those used in the polynomial module, do not follow
  this rule for efficiency reasons.} should be subclasses of the base class
\texttt{Basic}, which defines some basic methods for symbolic expression
trees. \texttt{Expr} is the subclass for mathematical expressions that can be
added and multiplied together. Instances of \texttt{Expr} typically represent
complex numbers, but may also include other ``rings'' like matrix expressions.
Not all SymPy classes are subclasses of \texttt{Expr}. For instance, logic expressions such
as \verb|And(x, y)| are subclasses of \texttt{Basic} but not of \texttt{Expr}.

The \texttt{Function} class is a subclass of \texttt{Expr} which makes it
easier to define mathematical functions called with arguments. This includes
named functions like $\sin(x)$ and $\log(x)$ as well as undefined functions
like $f(x)$. Subclasses of \texttt{Function} should define a
class method \texttt{eval}, which returns values for which the function should
be automatically evaluated, and \texttt{None} for arguments that should not be
automatically evaluated.

Many SymPy functions perform various evaluations down the expression tree.
Classes define their behavior in such functions by defining a relevant
\verb|_eval_|\texttt{\textit{*}} method. For instance, an object can indicate
to the \texttt{diff} function how to take the derivative of itself by defining
the \verb|_eval_derivative(self, x)| method, which may in turn call
\texttt{diff} on its \texttt{args}. The most common
\verb|_eval_|\texttt{\textit{*}} methods relate to the assumptions.
\verb|_eval_is_|\texttt{\textit{assumption}} defines the assumptions for
\textit{assumption}.

As an example of the notions presented in this section,
Listing~\ref{fig:gamma-example} presents a stripped down version of the gamma
function $\Gamma(x)$ from SymPy, which evaluates itself on positive integer
arguments, has the positive and real assumptions defined, can be rewritten in
terms of factorial with \verb|gamma(x).rewrite(factorial)|, and can be
differentiated. \texttt{fdiff} is a convenience method for subclasses of
\texttt{Function}. \texttt{fdiff} returns the derivative of the function
without considering the chain rule. \texttt{self.func} is used throughout
instead of referencing \texttt{gamma} explicitly so that potential subclasses
of \texttt{gamma} can reuse the methods.

\lstset{
  basicstyle=\ttfamily,
}

\begin{lstlisting}[caption={A stripped down version of \texttt{sympy.gamma}.},label=fig:gamma-example]
from sympy import Integer, Function, floor, factorial, polygamma

class gamma(Function)
    @classmethod
    def eval(cls, arg):
        if isinstance(arg, Integer) and arg.is_positive:
            return factorial(arg - 1)

    def _eval_is_positive(self):
        x = self.args[0]
        if x.is_positive:
            return True
        elif x.is_noninteger:
            return floor(x).is_even

    def _eval_is_real(self):
        x = self.args[0]
        # noninteger means real and not integer
        if x.is_positive or x.is_noninteger:
            return True

    def _eval_rewrite_as_factorial(self, z):
        return factorial(z - 1)

    def fdiff(self, argindex=1):
        from sympy.core.function import ArgumentIndexError
        if argindex == 1:
            return self.func(self.args[0])*polygamma(0, self.args[0])
        else:
            raise ArgumentIndexError(self, argindex)
\end{lstlisting}
The actual gamma function defined in SymPy has many more capabilities, such as
evaluation at rational points and series expansion.


\section{Features}
\label{sec:features}

%% List of Features and how to use
%%
%% Quick overview of the main modules, what it can do and so on. It should probably provide examples how to use sympy.
%%
%% See also the supplement (below)

% Features to discuss in-depth:

Although SymPy's extensive feature set cannot be covered in-depth in this paper,
calculus and other bedrock areas are discussed in their own subsections.
Additionally,
Table~\ref{features-table} gives a compact listing of all major capabilities
present in the SymPy codebase. This grants a sampling from the breadth of
topics and application domains that SymPy services. Unless stated otherwise,
all features noted in Table~\ref{features-table} are symbolic in nature.
Numeric features are discussed in Section~\ref{sec:numerics}.

\begin{longtable}[htbc]{p{0.20\linewidth}p{0.73\linewidth}}
\caption{SymPy Features and Descriptions\label{features-table}}\\
\toprule
\textbf{Feature} & \textbf{Description} \\
\midrule
Calculus & Algorithms for computing derivatives, integrals, and limits.\\

Category Theory & Representation of objects, morphisms, and diagrams. Tools
for drawing diagrams with Xy-pic.\\

Code Generation & Generation of compilable and executable code in a
variety of different programming languages from expressions directly. Target
languages include C, Fortran, Julia, JavaScript,
Mathematica, MATLAB and Octave, Python, and Theano.\\

Combinatorics \& Group Theory & Permutations, combinations,
partitions, subsets, various permutation groups (such as polyhedral, Rubik,
symmetric, and others), Gray codes~\cite{Nijenhuis1978combinatorial},
and Prufer sequences~\cite{biggs1976graph}.\\

Concrete Math & Summation, products, tools for determining whether summation
and product expressions are convergent, absolutely convergent, hypergeometric,
and for determining other properties; computation of Gosper's normal form~\cite{petkovvsek1996bak} for two univariate polynomials.\\

Cryptography & Block and stream ciphers, including shift, Affine,
substitution, Vigen\`{e}re's, Hill's, bifid, RSA, Kid RSA,
linear-feedback shift registers, and Elgamal encryption.\\

Differential Geometry & Representations of manifolds, metrics, tensor
products, and coordinate systems in Riemannian and pseudo-Riemannian
geometries~\cite{FunctionalDifferentialGeometry}.\\
% TODO: Someone verify that this is a good summary of the diffgeom module

Geometry & Representations of 2D geometrical entities, such as lines and
circles. Enables queries on these entities, such as asking the area of an
ellipse, checking for collinearity of a set of
points, or finding the intersection between objects.\\

Lie Algebras & Representations of Lie algebras and root systems.\\

Logic & Boolean expressions, equivalence testing, satisfiability, and normal
forms.\\

Matrices & Tools for creating matrices of symbols and expressions.
Both sparse and dense representations, as well as symbolic linear
algebraic operations (e.g., inversion and factorization), are
supported.\\

Matrix Expressions & Matrices with symbolic dimensions (unspecified entries).
Block matrices.\\

Number Theory & Prime number generation, primality testing, integer
factorization, continued fractions, Egyptian fractions, modular arithmetic,
quadratic residues, partitions, binomial and multinomial coefficients,
prime number tools, hexidecimal digits of $\pi$, and integer factorization. \\

Plotting & Hooks for visualizing expressions via matplotlib~\cite{Hunter:2007}
or as text drawings when lacking a graphical back-end. 2D function plotting,
3D function
plotting, and 2D implicit function plotting are supported.\\

Polynomials & Polynomial algebras over various coefficient domains.
Functionality ranges from simple operations (e.g., polynomial division) to
advanced computations (e.g., Gr\"obner bases~\cite{adams1994introduction} and multivariate
factorization over algebraic number domains).\\

Printing & Functions for printing SymPy expressions in the terminal with ASCII
or Unicode characters and converting SymPy expressions to \LaTeX{} and
MathML.\\

Quantum Mechanics & Quantum states, bra--ket notation, operators, basis sets,
representations, tensor products, inner products, outer products, commutators,
anticommutators, and specific quantum system implementations.\\

Series & Series expansion, sequences, and limits of sequences.
This includes Taylor, Laurent, and Puiseux series as well as special series, such
as Fourier and formal power series.\\

Sets & Representations of empty, finite, and infinite sets. This includes
special sets such as for all natural, integer, and complex numbers. Operations
on sets such as union, intersection, Cartesian product, and building sets from
other sets are supported.\\

Simplification & Functions for manipulating and simplifying expressions.
Includes algorithms for simplifying hypergeometric functions, trigonometric
expressions, rational functions, combinatorial functions, square root
denesting, and common subexpression elimination.\\

Solvers & Functions for symbolically solving equations, systems
of equations, both linear and non-linear, inequalities, ordinary differential
equations, partial differential equations, Diophantine equations, and
recurrence relations.\\

Special Functions & Implementations of a number of well known special functions,
including Dirac delta, Gamma, Beta, Gauss error functions, Fresnel integrals,
Exponential integrals, Logarithmic integrals, Trigonometric integrals, Bessel,
Hankel, Airy, B-spline, Riemann Zeta, Dirichlet eta, polylogarithm, Lerch
transcendent, hypergeometric, elliptic integrals, Mathieu, Jacobi polynomials,
Gegenbauer polynomial, Chebyshev polynomial, Legendre polynomial, Hermite
polynomial, Laguerre polynomial, and
spherical harmonic functions.\\

Statistics & Support for a random variable type as well as the ability to
declare this variable from prebuilt distribution functions such as
Normal, Exponential, Coin, Die, and other custom distributions~\cite{StatsMRocklin}.\\

Tensors & Symbolic manipulation of indexed objects.\\

Vectors & Basic operations on vectors and differential calculus with respect
to 3D Cartesian coordinate systems.\\
\bottomrule

\end{longtable}

\subsection{Simplification}


% polynomial expressions

% functions

% expand( ), factor( ), collect( ), together( ), apart( )
%% maybe a table best suits this part.

% simplification: simplify, sqrt denest, fu, trigsimp

The generic way to simplify an expression is by calling the \texttt{simplify}
function.
It must be emphasized that simplification is not an unambigously defined
mathematical operation~\cite{Carette2004understanding}.
The \texttt{simplify} function applies several simplification routines along
with some heuristics to make the output expression as ``simple'' as possible.

It is often preferable to apply more directed simplification functions. These
apply very specific rules to the input expression and are often able to make
guarantees about the output (for instance, the \texttt{factor} function, given
a polynomial with rational coefficients in several variables, is guaranteed to
produce a factorization into irreducible factors). Table~\ref{simplify-table}
lists some common simplification functions.

% TODO: remove the empty lines between rows
\begin{longtable}[htbc]{lp{0.83\linewidth}}
\caption{Some SymPy Simplification Functions\label{simplify-table}}\\
\toprule
\verb|expand| & expand the expression
\begin{verbatim}
>>> expand((x + y)**3)
x**3 + 3*x**2*y + 3*x*y**2 + y**3
\end{verbatim}
\\
\verb|factor| & factor a polynomial into irreducibles
\begin{verbatim}
>>> factor(x**3 + 3*x**2*y + 3*x*y**2 + y**3)
(x + y)**3
\end{verbatim}
\\
\verb|collect| & collect polynomial coefficients
\begin{verbatim}
>>> collect(y*x**2 + 3*x**2 - x*y + x - 1, x)
x**2*(y + 3) + x*(-y + 1) - 1
\end{verbatim}
\\
\verb|cancel| & rewrite a rational function as $p/q$ with common factors
canceled
\begin{verbatim}
>>> cancel((x**2 + 2*x + 1)/(x**2 - 1))
(x + 1)/(x - 1)
\end{verbatim}
\\
\verb|apart| & compute the partial fraction decomposition of a rational function
\begin{verbatim}
>>> apart((x**3 + 4*x - 1)/(x**2 - 1))
x + 3/(x + 1) + 2/(x - 1)
\end{verbatim}
\\
\verb|trigsimp| & simplify trigonometric expressions~\cite{fu2006automated}
\begin{verbatim}
>>> trigsimp(cos(x)**2*tan(x) - sin(2*x))
-sin(2*x)/2
\end{verbatim}
\\
\bottomrule
\end{longtable}

Substitutions are performed through the \texttt{.subs} method.
\begin{verbatim}
>>> (sin(x) + x**2 + 1).subs(x, y + 1)
(y + 1)**2 + sin(y + 1) + 1
\end{verbatim}


\subsection{Calculus}
\label{sec:calculus}
SymPy provides all the basic operations of calculus, such as calculating
limits, derivatives, integrals, or summations.

Limits are computed with the \verb|limit| function, using the Gruntz
algorithm~\cite{Gruntz1996limits} for computing symbolic limits and heuristics (a description of the Gruntz algorithm may be found in the supplement).
For example, the following computes
$\lim\limits_{x\to \infty} x\sin(\frac{1}{x})=1$. Note that SymPy denotes
$\infty$ as \verb|oo|.
\begin{verbatim}
>>> limit(x*sin(1/x), x, oo)
1
\end{verbatim}
As a more complex example, SymPy computes \[\lim\limits_{x\to 0}{\left(2 e^{\frac{1 - \cos{\left (x \right )}}{\sin{\left (x \right )}}} -
  1\right)}^{\frac{\sinh{\left (x \right )}}{\operatorname{atan}^{2}{\left (x
      \right )}}} = e.\]
\begin{verbatim}
>>> limit((2*E**((1-cos(x))/sin(x))-1)**(sinh(x)/atan(x)**2), x, 0)
E
\end{verbatim}

Derivatives are computed with the \verb|diff| function, which recursively uses
the various differentiation rules.
\begin{verbatim}
>>> diff(sin(x)*exp(x), x)
exp(x)*sin(x) + exp(x)*cos(x)
\end{verbatim}

Integrals are calculated with the \verb|integrate| function. SymPy
implements a combination of the Risch
algorithm~\cite{bronstein2005integration}, table lookups, a reimplementation
of Manuel Bronstein's ``Poor Man's Integrator''~\cite{Bronstein2005pmint}, and
an algorithm for computing integrals based on Meijer G-functions~\cite{Roach1996hyper,roach1997meijerg}. These allow
SymPy to compute a wide variety of indefinite and definite integrals. The
Meijer G-function algorithm and the Risch algorithm are respectively
demonstrated below by the computation of \[\int_{0}^{\infty} e^{-s t}\log{\left (t \right )}\, dt = - \frac{ \log{\left (s \right )} + \gamma}{s}\] and \[\int \frac{- 2 x^{2} \left(\log{\left (x \right )} + 1\right) e^{x^{2}} + {\left(e^{x^{2}} + 1\right)}^{2}}{x {\left(e^{x^{2}} + 1\right)}^{2} \left(\log{\left (x \right )} + 1\right)}\, dx = \log{\left (\log{\left (x \right )} + 1 \right )} + \frac{1}{e^{x^{2}} + 1}.\]
\begin{verbatim}
>>> s, t = symbols('s t', positive=True)
>>> integrate(exp(-s*t)*log(t), (t, 0, oo)).simplify()
-(log(s) + EulerGamma)/s
>>> integrate((-2*x**2*(log(x) + 1)*exp(x**2) +
... (exp(x**2) + 1)**2)/(x*(exp(x**2) + 1)**2*(log(x) + 1)), x)
log(log(x) + 1) + 1/(exp(x**2) + 1)
\end{verbatim}

Summations are computed with \verb|summation|  using a combination of Gosper's
algorithm~\cite{gosper1978decision}, an algorithm that uses Meijer
G-functions~\cite{Roach1996hyper,roach1997meijerg}, and heuristics. Products
are computed with \verb|product| function via a suite of heuristics.
% TODO: Are there other summation algorithms implemented?
% TODO: A good summation example or two
\begin{verbatim}
>>> i, n = symbols('i n')
>>> summation(2**i, (i, 0, n - 1))
2**n - 1
>>> summation(i*factorial(i), (i, 1, n))
n*factorial(n) + factorial(n) - 1
\end{verbatim}

Integrals, derivatives, summations, products, and limits that cannot be
computed return unevaluated objects. These can also be created directly if the
user chooses.
\begin{verbatim}
>>> integrate(x**x, x)
Integral(x**x, x)
>>> Sum(2**i, (i, 0, n - 1))
Sum(2**i, (i, 0, n - 1))
\end{verbatim}


\subsection{Polynomials}
% Polynomials

SymPy implements a suite of algorithms for polynomial manipulation,
which ranges from relatively simple algorithms for doing arithmetic of
polynomials, to advanced methods for factoring multivariate polynomials
into irreducibles, symbolically determining real and complex root isolation
intervals, or computing Gr\"{o}bner bases.

Polynomial manipulation is useful on its own. Within SymPy, though, it is mostly used
indirectly as a tool in other areas of the library. In fact, many mathematical
problems in symbolic computing are first expressed using entities from the
symbolic core, preprocessed, and then transformed into a problem in the
polynomial algebra, where generic and efficient algorithms are used to solve
the problem and solutions to the original problem are recovered.
This is a common scheme in symbolic integration or summation algorithms.

SymPy implements dense and sparse polynomial representations. Both are used in
the univariate and multivariate cases. The dense representation is the default
for univariate polynomials. For multivariate polynomials, the choice of
representation is based on the application. The most common case for the sparse
representation is algorithms for computing Gr\"{o}bner bases (Buchberger, F4,
and F5).
% TODO: citation
This is
because different monomial orderings can be expressed easily in this
representation. However, algorithms for computing multivariate GCDs or
factorizations, at least those currently implemented in SymPy,
% TODO: citation
are better expressed when the representation is dense. The dense multivariate
representation is specifically a recursively dense representation, where
polynomials in $K[x_0, x_1, \dotsc, x_n]$ are viewed as a polynomials in
$K[x_0][x_1]\dotso[x_n]$. Note that despite this, the coefficient domain $K$,
can be a multivariate polynomial domain as well. The dense recursive
representation in Python gets inefficient when the number of variables gets
high.

Some examples for the \texttt{sympy.polys} module can be found in the
supplement.


\subsection{Printers}

SymPy has a rich collection of expression printers.
By default, an interactive Python session will render the
\verb|str| form of an expression, which has been used in all the examples in
this paper so far. The \verb|str| form of an expression is valid Python and
roughly matches what a user would type to enter the expression.

\begin{verbatim}
>>> phi0 = Symbol('phi0')
>>> str(Integral(sqrt(phi0), phi0))
'Integral(sqrt(phi0), phi0)'
\end{verbatim}

Expressions can be printed in 2D with monospace fonts via \verb|pprint|.
Unicode characters are used for rendering mathematical symbols such as integral signs,
square roots, and parentheses. Greek letters and subscripts in symbol names
that have Unicode code points associated
are also rendered automatically.
% TODO cite unicode?

\noindent
\includegraphics[width=1\textwidth]{pprint.pdf}
Alternately, the \verb|use_unicode=False| flag can be set, which causes the
expression to be printed using only ASCII characters.
% TODO cite ASCII

\begin{verbatim}
>>> pprint(Integral(sqrt(phi0 + 1), phi0), use_unicode=False)
  /
 |
 |   __________
 | \/ phi0 + 1  d(phi0)
 |
/
\end{verbatim}

The function \verb|latex| returns a \LaTeX{} representation of an expression.
% TODO cite latex

\begin{verbatim}
>>> print(latex(Integral(sqrt(phi0 + 1), phi0)))
\int \sqrt{\phi_{0} + 1}\, d\phi_{0}
\end{verbatim}

Users are encouraged to run the \verb|init_printing| function at the beginning
of interactive sessions, which automatically enables the best pretty printing
supported by their environment. In the Jupyter Notebook or Qt
Console~\cite{perez2007ipython}, the \LaTeX{} printer is used to render
expressions using MathJax or \LaTeX{}, if it is installed on the system. The
2D text representation is used otherwise.

Other printers such as MathML are also available. SymPy uses an extensible
printer subsystem for customizing any given
printer, and allows custom objects to define their printing behavior for any
printer. The code generation functionality of SymPy
relies on this subsystem to convert expressions into code in various target
programming languages.


% Solvers (regular equations, maybe also mention other types of solvers like ODEs/recurrence/Diophantine)
\subsection{Solvers}
SymPy has a module of equation solvers for symbolic equations. There are two
functions for solving algebraic equations in SymPy: \texttt{solve}
and \texttt{solveset}.
\texttt{solveset} has several design changes with respect to the older
\texttt{solve} function. This distinction is present in order to resolve the
usability issues with the
previous \texttt{solve} function API while maintaining backward compatibility
with earlier versions of SymPy.
\texttt{solveset} only requires the necessary input information from the user.
The function signatures of \texttt{solve} and \texttt{solveset} are
\begin{verbatim}
solve(f, *symbols, **flags)
solveset(f, symbol, domain=S.Complexes)
\end{verbatim}
The \texttt{domain} parameter is typically either \texttt{S.Complexes} (the
default) or \texttt{S.Reals}, which causes it to only return real solutions.

Additionally, \texttt{solve} has an inconsistent output API for various types
of inputs. For instance, depending on the input, sometimes it returns a Python
list and sometimes it returns a Python dictionary. On the other hand, the
\texttt{solveset} has a canonical output API.\ \texttt{solveset} always returns
a SymPy set object.

Both functions implicitly assume that expressions are equal to 0. For
instance, \texttt{solveset(x - 1, x)} solves $x - 1 = 0$ for $x$.

\texttt{solveset} is under active development as a planned replacement for
\texttt{solve}. There are certain features which are implemented in
\texttt{solve} that are not yet implemented in \texttt{solveset}. Notably,
these include nonlinear multivariate system and transcendental equations.

More examples of \texttt{solveset} and \texttt{solve} can be found in the
supplement.


% Matrices (worth emphasizing that they are symbolic)
\subsection{Matrices}

Computation on matrices with symbolic entries is important for many algorithms
within SymPy, as well as being an important feature in its own right.
\begin{verbatim}
>>> A = Matrix(2, 2, [x, x + y, y, x])
>>> A
Matrix([
[x, x + y],
[y,     x]])
\end{verbatim}

SymPy matrices support common symbolic linear algebra manipulations, including
matrix addition, multiplication, exponentiation, computing determinants,
solving linear systems, and computing inverses using LU decomposition, LDL
decomposition, Gauss-Jordan elimination, Cholesky decomposition, Moore-Penrose
pseudoinverse, and adjugate matrix.

All operations are computed symbolically. For instance, eigenvalues are computed
by generating the characteristic polynomial using the Berkowitz algorithm and
then solving it using polynomial routines.

\begin{verbatim}
>>> A.eigenvals()
{x - sqrt(y*(x + y)): 1, x + sqrt(y*(x + y)): 1}
\end{verbatim}

Internally these matrices store the elements as a list of lists (LIL), making
it a dense representation.\footnote{Similar to the polynomials module, dense
  here means that all entries are stored in memory, contrasted with a sparse
  representation where only nonzero entries are stored.} For storing sparse
matrices, the \verb|SparseMatrix| class can be used. Sparse matrices store the
elements in a dictionary of keys (DOK) format.

SymPy also supports matrices with symbolic dimension values. \verb|MatrixSymbol|
represents a matrix with dimensions $m\times n$, where $m$ and $n$ can be
symbolic. Matrix addition and multiplication, scalar operations, matrix inverse,
and transpose are stored symbolically as matrix expressions.

Block matrices are also implemented in SymPy. \verb|BlockMatrix| elements can be any
matrix expression which includes explicit matrices, matrix symbols, and block
matrices. All functionalities of matrix expressions are also present in
\verb|BlockMatrix|.

When symbolic matrices are combined with the assumptions module for logical
inference they provide powerful reasoning over invertibility,
semi-definiteness, orthogonality, etc. which are valuable in the construction
of numerical linear algebra systems.

More examples for \verb|Matrix| and \verb|BlockMatrix| can be found in the
supplement.




\section{Numerics}
\label{sec:numerics}

%% Description of some algorithms (example: integration with Risch, Meijer G, Gruntz, polys)
%%
%% Description of numerics/mpmath (Fredrik)

Floating point numbers in SymPy are implemented by the \texttt{Float} class,
which represents an arbitrary-precision binary floating-point number by
storing its value and precision (in bits). This representation is distinct
from the Python built-in \texttt{float} type, which is a wrapper around
machine \texttt{double} types and uses a fixed precision (53-bit).

Because Python \texttt{float} literals are limited in precision, strings
should be used to input precise decimal values:
\begin{verbatim}
>>> Float(1.1)
1.10000000000000
>>> Float(1.1, 30)   # precision equivalent to 30 digits
1.10000000000000008881784197001
>>> Float("1.1", 30)
1.10000000000000000000000000000
\end{verbatim}
The \texttt{evalf} method converts a constant symbolic expression to a
\texttt{Float} with the specified precision, here 25 digits:
\begin{verbatim}
>>> (pi + 1).evalf(25)
4.141592653589793238462643
\end{verbatim}
\texttt{Float} numbers do not track their accuracy,
and should be used with caution within symbolic expressions
since familiar dangers of floating-point arithmetic apply~\cite{goldberg1991every}.
A notorious case is that of catastrophic cancellation:
\begin{verbatim}
>>> cos(exp(-100)).evalf(25) - 1
0
\end{verbatim}
Applying the \texttt{evalf} method to the whole expression solves
this problem. Internally, \texttt{evalf} estimates the number of accurate
bits of the floating-point
approximation for each sub-expression, and adaptively increases the
working precision until the estimated accuracy of the
final result matches the sought number of decimal digits:
\begin{verbatim}
>>> (cos(exp(-100)) - 1).evalf(25)
-6.919482633683687653243407e-88
\end{verbatim}
The \texttt{evalf} method works with complex numbers and supports
more complicated expressions, such as
special functions, infinite series, and integrals.
The internal error tracking does not provide rigorous error bounds
(in the sense of interval arithmetic) and cannot be used to accurately track
uncertainty in measurement data;
the sole purpose is to mitigate loss of accuracy that typically occurs
when converting symbolic expressions to numerical values.

\subsection{The mpmath library}

The implementation of arbitrary-precision floating-point arithmetic is
supplied by the mpmath library. Originally, it was developed as a SymPy
module but has subsequently been moved to a standalone pure-Python package.
The basic datatypes in mpmath are \texttt{mpf} and \texttt{mpc}, which
respectively act as multiprecision substitutes for Python's \texttt{float} and
\texttt{complex}. The floating-point precision is controlled by a global
context:

% doctest printer doesn't display "mpf"
% no-doctest
\begin{verbatim}
>>> import mpmath
>>> mpmath.mp.dps = 30    # 30 digits of precision
>>> mpmath.mpf("0.1") + mpmath.exp(-50)
mpf('0.100000000000000000000192874984794')
>>> print(_)   # pretty-printed
0.100000000000000000000192874985
\end{verbatim}

For pure numerical computing, it is convenient to use mpmath directly
with \texttt{from mpmath import *}.  Nevertheless, it is best to avoid such an
import statement when using SymPy simultaneously, since the names of numerical
functions such as \texttt{exp} will collide the symbolic counterparts
in SymPy.

Like SymPy, mpmath is a pure Python library.
Internally, mpmath represents a floating-point number
${(-1)}^s x \cdot 2^y$ by a tuple $(s, x, y, b)$ where
$x$ and $y$ are arbitrary-size Python integers
and the redundant integer $b$ stores the bit length of $x$ for quick access.
If GMPY~\cite{GMPY} is installed, mpmath automatically uses
the \texttt{gmpy.mpz} type for~$x$, and GMPY methods
for rounding-related operations, improving performance.

The mpmath library supports
special functions, root-finding, linear algebra, polynomial approximation,
and numerical computation of limits, derivatives, integrals, infinite
series, and ODE solutions. All features work in arbitrary precision
and use algorithms that allow computing hundreds of digits rapidly
(except in degenerate cases).

The double exponential (tanh-sinh) quadrature is used for numerical
integration by default. For smooth integrands, this algorithm usually
converges extremely rapidly, even when the integration interval is infinite
or singularities are present at the endpoints~\cite{takahasi1974double,bailey2005comparison}.
However, for good performance, singularities
in the middle of the interval must be specified
by the user.
To evaluate slowly converging limits and infinite series, mpmath
automatically tries Richardson extrapolation and the
Shanks transformation
(Euler-Maclaurin summation can also be used)~\cite{BenderOrszag1999}.
A function to evaluate oscillatory integrals by means of convergence
acceleration is also available.

A wide array of higher mathematical functions are implemented
with full support for complex values of all parameters and arguments,
including complete and incomplete gamma functions,
Bessel functions, orthogonal polynomials, elliptic functions and integrals,
zeta and polylogarithm functions,
the generalized hypergeometric function, and the Meijer G-function.
The Meijer G-function instance
$G_{1, 3}^{3, 0}\left(0 ; \tfrac{1}{2}, -1, - \tfrac{3}{2} | x \right)$
is a good test case~\cite{Toth2007}; past versions of both Maple and
Mathematica produced incorrect numerical values for large $x > 0$.
Here, mpmath automatically removes an internal singularity
and compensates for cancellations (amounting to 656 bits
of precision when $x = 10000$), giving correct values:
% doctest printer doesn't display "mpf"
% no-doctest
\begin{verbatim}
>>> mpmath.mp.dps = 15
>>> mpmath.meijerg([[],[0]],[[-0.5,-1,-1.5],[]],10000)
mpf('2.4392576907199564e-94')
\end{verbatim}

Equivalently, with SymPy's interface this function can be evaluated as:
\begin{verbatim}
>>> meijerg([[],[0]],[[-S(1)/2,-1,-S(3)/2],[]],10000).evalf()
2.43925769071996e-94
\end{verbatim}

Symbolic integration and summation often produces hypergeometric
and Meijer G-function closed forms (see Subsection~\ref{sec:calculus});
numerical evaluation of such special functions is a useful complement
to direct numerical integration and summation.



\section{Domain Specific Submodules}
\label{sec:domain_specific}

SymPy includes several packages that allow users to solve domain specific
problems. For example, a comprehensive physics package is included that is
useful for solving problems in mechanics, optics, and quantum
mechanics along with support for manipulating physical quantities with units.


\subsection{Classical Mechanics}
One of the core domains that SymPy suports is the physics of classical
mechanics. This is in turn separated into two distinct components:
vector algebra symbolics and mechanics.

\subsubsection{Vector Algebra}
% TODO This section requires some citations.

The \verb|sympy.physics.vector| package provides reference frame-, time-, and
space-aware vector and dyadic objects that allow for three-di\-men\-sional
operations such as addition, subtraction, scalar multiplication, inner and
outer products, and cross products. Both of these objects can be written in
very compact notation that make it easy to express the vectors and dyadics in
terms of multiple reference frames with arbitrarily defined relative
orientations. The vectors are used to specify the positions, velocities, and
accelerations of points; orientations, angular velocities, and angular
accelerations of reference frames; and forces and torques. The dyadics are
essentially reference frame-aware $3 \times 3$ tensors~\cite{tai1997generalized}.
The vector and dyadic
objects can be used for any one-, two-, or three-dimensional vector algebra, and
they provide a strong framework for building physics and engineering tools.

The following Python code demonstrates how a vector is created using
the orthogonal unit vectors of three reference frames that are oriented with
respect to each other, and the result of expressing the vector in the $A$
frame. The $B$ frame is oriented with respect to the $A$ frame using Z-X-Z
Euler Angles of magnitude $\pi$, $\frac{\pi}{2}$, and
$\frac{\pi}{3}$\si{\radian}, respectively, whereas the $C$ frame is oriented
with respect to the $B$ frame through a simple rotation about the $B$ frame's
$X$ unit vector through $\frac{\pi}{2}$\si{\radian}.

\begin{verbatim}
>>> from sympy.physics.vector import ReferenceFrame
>>> A = ReferenceFrame('A')
>>> B = ReferenceFrame('B')
>>> C = ReferenceFrame('C')
>>> B.orient(A, 'body', (pi, pi/3, pi/4), 'zxz')
>>> C.orient(B, 'axis', (pi/2, B.x))
>>> v = 1*A.x + 2*B.z + 3*C.y
>>> v
A.x + 2*B.z + 3*C.y
>>> v.express(A)
A.x + 5*sqrt(3)/2*A.y + 5/2*A.z
\end{verbatim}

\subsubsection{Mechanics}

The \verb|sympy.physics.mechanics| package utilizes the \texttt{sympy.\allowbreak{}physics.\allowbreak{}vector} package
to populate time-aware particle and rigid-body objects to fully describe the
kinematics and kinetics of a rigid multi-body system. These objects store all
of the information needed to derive the ordinary differential or differential
algebraic equations that govern the motion of the system, i.e., the equations
of motion. These equations of motion abide by Newton's laws of motion and can
handle arbitrary kinematic constraints or complex loads. The package
offers two automated methods for formulating the equations of motion based on
Lagrangian Dynamics~\cite{Lagrange1811} and Kane's Method~\cite{Kane1985}.
Lastly, there are automated linearization routines for constrained dynamical
systems~\cite{Peterson2014}.

\subsection{Quantum Mechanics}

The \verb|sympy.physics.quantum| package has extensive capabilities for
performing symbolic quantum mechanics, using Python objects to represent the
different mathematical objects relevant in quantum theory~\cite{Sakurai2010}:
states (bras and kets), operators (unitary, Hermitian, etc.), and basis sets, as
well as operations on these objects such as representations, tensor products,
inner products, outer products, commutators, and anticommutators. The base
objects are designed in the most general way possible to enable any particular
quantum system to be implemented by subclassing the base operators and defining
the relevant class methods to provide system-specific logic.

Symbolic quantum operators and states may be defined, and one can perform
a full range of operations with them.
\begin{verbatim}
>>> from sympy.physics.quantum import Commutator, Dagger, Operator
>>> from sympy.physics.quantum import Ket, qapply
>>> A = Operator('A')
>>> B = Operator('B')
>>> C = Operator('C')
>>> D = Operator('D')
>>> a = Ket('a')
>>> comm = Commutator(A, B)
>>> comm
[A,B]
>>> qapply(Dagger(comm*a)).doit()
-<a|*(Dagger(A)*Dagger(B) - Dagger(B)*Dagger(A))
\end{verbatim}
Commutators can be expanded using common commutator identities:
\begin{verbatim}
>>> Commutator(C+B, A*D).expand(commutator=True)
-[A,B]*D - [A,C]*D + A*[B,D] + A*[C,D]
\end{verbatim}

On top of this set of base objects, a number of specific quantum systems have
been implemented in a fully symbolic framework. These include:

\begin{itemize}

\item Many of the exactly solvable quantum systems, including simple harmonic
oscillator states and raising/lowering operators, infinite square well states,
and 3D position and momentum operators and states.

\item Second quantized formalism of non-relativistic many-body quantum
mechanics~\cite{FetterWalecka2003}.

\item Quantum angular momentum~\cite{Zare1991}. Spin operators and their
eigenstates can be represented in any basis and for any quantum numbers.
A rotation operator representing the Wigner-D matrix, which may be defined
symbolically or numerically, is also implemented to rotate spin eigenstates.
Functionality for coupling and uncoupling of arbitrary spin eigenstates is
provided, including symbolic representations of Clebsch-Gordon coefficients and
Wigner symbols.

\item Quantum information and computing~\cite{Nielsen2011}. Multidimensional
qubit states, and a full set of one- and two-qubit gates are provided and can
be represented symbolically or as matrices/vectors. With these building blocks,
it is possible to implement a number of basic quantum algorithms including the
quantum Fourier transform, quantum error correction, quantum teleportation,
Grover's algorithm, dense coding, etc. In addition, any quantum circuit may be
plotted using the \verb|circuit_plot| function (Figure~\ref{fig-circuitplot-qft}).


\end{itemize}

Here are a few short examples of the quantum information and computing capabilities
in \verb|sympy.physics.quantum|. Start with a simple four-qubit state and flip the second
qubit from the right using a Pauli-X gate:

\begin{verbatim}
>>> from sympy.physics.quantum.qubit import Qubit
>>> from sympy.physics.quantum.gate import XGate
>>> q = Qubit('0101')
>>> q
|0101>
>>> X = XGate(1)
>>> qapply(X*q)
|0111>
\end{verbatim}
Qubit states can also be used in adjoint operations, tensor products, inner/outer
products:
\begin{verbatim}
>>> Dagger(q)
<0101|
>>> ip = Dagger(q)*q
>>> ip
<0101|0101>
>>> ip.doit()
1
\end{verbatim}
Quantum gates (unitary operators) can be applied to transform these states and
then classical measurements can be performed on the results:
\begin{verbatim}
>>> from sympy.physics.quantum.qubit import measure_all
>>> from sympy.physics.quantum.gate import H, X, Y, Z
>>> c = H(0)*H(1)*Qubit('00')
>>> c
H(0)*H(1)*|00>
>>> q = qapply(c)
>>> measure_all(q)
[(|00>, 1/4), (|01>, 1/4), (|10>, 1/4), (|11>, 1/4)]
\end{verbatim}
\begin{figure}[htbp]
\begin{center}
\includegraphics[scale=0.65]{images/circuitplot-qft}
\caption{The circuit diagram for a three-qubit quantum Fourier transform
generated by SymPy.}
\label{fig-circuitplot-qft}
\end{center}
\end{figure}
Lastly, the following example demonstrates creating a three-qubit quantum Fourier
transform, decomposing it into one- and two-qubit gates, and then generating a
circuit plot for the sequence of gates (see Figure~\ref{fig-circuitplot-qft}).
\begin{verbatim}
>>> from sympy.physics.quantum.qft import QFT
>>> from sympy.physics.quantum.circuitplot import circuit_plot
>>> fourier = QFT(0,3).decompose()
>>> fourier
SWAP(0,2)*H(0)*C((0),S(1))*H(1)*C((0),T(2))*C((1),S(2))*H(2)
>>> c = circuit_plot(fourier, nqubits=3)
\end{verbatim}


\section{Conclusion and future work}
\label{sec:conclusion}

SymPy is a robust computer algebra system that provides a wide array of
features both in traditional computer algebra and in broad scientific
disciplines. It is written in the general purpose Python language
which allows it to be used in a first-class way with other Python projects,
including the scientific Python stack. SymPy is designed to be used in an
extensible way and, unlike many other CASs, both as an end-user application and
as a library.

SymPy expressions are immutable trees of Python objects. SymPy uses Python both
as the internal language and the user language, meaning users can use the same
methods that the library implements to extend it. SymPy has an assumptions
system for declaring and deducing mathematical properties on expressions.

SymPy has submodules for many areas of mathematics. It has functions for
simplifying expressions, doing common calculus operations, pretty printing
expressions, solving equations, and symbolic matrices. Other included areas
are discrete math, concrete math, plotting, geometry, statistics,
polynomials, sets, series, vectors, combinatorics, group theory, code
generation, tensors, Lie algebras, cryptography, and special functions.
Additionally, SymPy contains submodules targeting certain specific domains,
such as classical mechanics and quantum mechanics.  This breadth of domains is
due to a strong and vibrant user community that were attracted to SymPy because
of its ease of access.

% Future work:

Some of the planned future work for SymPy includes work on improving code
generation, improvements to the speed of SymPy, improving the assumptions
system, and improving the solvers module.

% TODO: Maybe one sentence for each item

Work is being done on an assumptions subsystem, distinct from the one
discussed in section~\ref{sec:assumptions}. The new system stores assumption
predicates separate from objects, and uses a SAT solver to do inference.

% Feel free to add stuff here.

% TODO: Mention SymEngine.


\section{Acknowledgements}
\label{sec:acknowledgements}

% People should add their funding agency acknowledgements here.
Sandia is a multiprogram laboratory operated by Sandia Corporation, a Lockheed Martin Company, for the United States Department of Energy's National Nuclear Security Administration under Contract DE-AC04-94AL85000.

The Google Summer of Code, an international annual program, in which Google awards stipends to all students who successfully complete a requested free and open-source software coding project during the summer.


\bibliography{paper}

\end{document}
