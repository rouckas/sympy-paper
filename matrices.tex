SymPy supports matrices with symbolic expressions as elements.

\begin{verbatim}
>>> A = Matrix(2, 2, [x, x + y, y, x])
>>> A
Matrix([
[x, x + y],
[y,     x]])
\end{verbatim}

All SymPy matrix types perform linear algebra including matrix addition,
multiplication, exponentiation, computing determinants, solving linear systems,
and computing inverses using LU decomposition, LDL decomposition, Gauss-Jordan
elimination, Cholesky decomposition, Moore-Penrose pseudoinverse, and adjugate
matrix.

All operations are computed symbolically. For instance, eigenvalues are computed
by generating the characteristic polynomial using the Berkowitz algorithm and
then solving it using polynomial routines. Diagonalizable matrices can be
diagonalized first to compute the eigenvalues.

\begin{verbatim}
>>> A.eigenvals()
{x - sqrt(y*(x + y)): 1, x + sqrt(y*(x + y)): 1}
\end{verbatim}

Internally these matrices store the elements as a list of lists, making it a dense
representation. For storing sparse matrices, the \verb|SparseMatrix| class can
be used. Sparse matrices store the elements in a dictionary of keys (DoK)
format.

SymPy also supports matrices with symbolic dimension values. \verb|MatrixSymbol|
represents a matrix with dimensions $m\times n$, where $m$ and $n$ can be
symbolic. Matrix addition and multiplication, scalar operations, matrix inverse,
and transpose are stored symbolically as matrix expressions.

Block matrices are also implemented in SymPy. \verb|BlockMatrix| elements can be any
matrix expression which includes explicit matrices, matrix symbols, and block
matrices. All functionalities of matrix expressions are also present in
\verb|BlockMatrix|.

When symbolic matrices are combined with the assumptions module for logical
inference they provide powerful reasoning over invertibility,
semi-definiteness, orthogonality, etc. which are valuable in the construction
of numerical linear algebra systems.
