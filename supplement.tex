\subsection{Limits: The Gruntz Algorithm}

\input{gruntz.tex}

% Series module (Formal Power Series, Fourier Series)
\subsection{Series}

\input{series.tex}

% Logic module
\subsection{Logic}


SymPy supports construction and manipulation of boolean expressions
through the \texttt{sympy.logic} module. SymPy symbols can be used as
propositional variables and also be substituted as \texttt{True}
or \texttt{False}. A good number of manipulation features for boolean
expressions have been implemented in the \texttt{sympy.logic} module.

\subsubsection{Constructing boolean expressions}

A boolean variable can be declared as a SymPy \verb|Symbol|. Python operators
\texttt{\&}, \texttt{\textbar{}} and \texttt{\textasciitilde{}} are overridden
when using SymPy objects to use the SymPy functionality for logical
\texttt{And}, \texttt{Or}, and \texttt{Not}. Other logic functions are also
integrated into SymPy, including \texttt{Xor} and \texttt{Implies}, which are
constructed with \texttt{\^{}} and \texttt{\textgreater{}\textgreater{}},
respectively. The above are just a shorthand, expressions can also be
constructed by directly creating the relevant objects: \verb|And()|,
\verb|Or()|, \verb|Not()|, \verb|Xor()|, \verb|Implies()|, \verb|Nand()|,
\verb|Nor()|, etc.

\begin{verbatim}
>>> from sympy import *
>>> x, y, z = symbols('x y z')
>>> e = (x & y) | z
>>> e.subs({x: True, y: True, z: False})
True
\end{verbatim}

\subsubsection{CNF and DNF}

Any boolean expression can be converted to conjunctive normal form, disjunctive
normal form, and negation normal form. The API also exposes methods to check if
a boolean expression is in any of the above mentioned forms.

\begin{verbatim}
>>> from sympy.logic.boolalg import is_dnf, is_cnf
>>> x, y, z = symbols('x y z')
>>> to_cnf((x & y) | z)
And(Or(x, z), Or(y, z))
>>> to_dnf(x & (y | z))
Or(And(x, y), And(x, z))
>>> is_cnf((x | y) & z)
True
>>> is_dnf((x & y) | z)
True
\end{verbatim}

\subsubsection{Simplification and Equivalence}

The module supports simplification of given boolean expression by making
deductions from the expression. Equivalence of two logical expressions can also
be checked. In the case of equivalence, it is possible to return the mapping of
variables in two expressions so as to represent the same logical behavior.

\begin{verbatim}
>>> from sympy import *
>>> a, b, c, x, y, z = symbols('a b c x y z')
>>> e = a & (~a | ~b) & (a | c)
>>> simplify(e)
And(Not(b), a)
>>> e1 = a & (b | c)
>>> e2 = (x & y) | (x & z)
>>> bool_map(e1, e2)
(And(Or(b, c), a), {a: x, b: y, c: z})
\end{verbatim}

\subsubsection{SAT solving}

The module also supports satisfiability (SAT) checking of a given boolean
expression. If satisfiable, it is possible to return a model for which the
expression is satisfiable. The API also supports returning all possible models.
The SAT solver has a clause learning DPLL algorithm implemented with a watch
literal scheme and VSIDS heuristic\cite{moskewicz2008method}.

\begin{verbatim}
>>> from sympy import *
>>> a, b, c = symbols('a b c')
>>> satisfiable(a & (~a | b) & (~b | c) & ~c)
False
>>> satisfiable(a & (~a | b) & (~b | c) & c)
{a: True, b: True, c: True}
\end{verbatim}


\subsection{Diophantine Equations}

\input{diophantine}

% Sets
\subsection{Sets}
%% Sets

SymPy supports representation of a wide variety of mathematical sets. This is
achieved by first defining abstract representations of atomic set classes and
then combining and transforming them using various set operations.

Each of the set classes inherits from the base class \texttt{Set} and defines
methods to check membership and calculate unions, intersections, and set
differences. When these methods are not able to evaluate to atomic set
classes, they are represented as abstract unevaluated objects.

SymPy has the following atomic set classes:

\begin{itemize}

    \item \verb|EmptySet| represents the empty set $\emptyset$.

    \item \verb|UniversalSet| is an abstract ``universal set'' of which
      everything is a member. The union of the universal set with any set
      gives the universal set and the intersection gives the other set itself.

    \item \verb|FiniteSet| is functionally equivalent to Python's built
      in \texttt{set} object. Its members can be any SymPy object including
      other sets.

    \item \verb|Integers| represents the set of integers $\mathbb{Z}$.

    \item \verb|Naturals| represents the set of natural numbers $\mathbb{N}$,
      i.e., the set of positive integers.

    \item \verb|Naturals0| represents the set of whole numbers $\mathbb{N}_0$,
      which are all the non-negative integers.

    \item \verb|Range| represents a range of integers. A range is defined by
      specifying a start value, an end value, and a step size. The enumeration
      of a \texttt{Range} object is functionally equivalent to Python's
      \texttt{range} except it supports infinite endpoints, allowing the
      representation of infinite ranges.

    \item \verb|Interval| represents an interval of real numbers. It is
      defined by giving the start and the end points and by specifying if the interval is open
      or closed on the respective ends.


\end{itemize}


%% Operations

Other than unevaluated classes of \texttt{Union}, \texttt{Intersection}, and
\texttt{Complement} operations, SymPy has the following set classes.

\begin{itemize}

    \item \verb|ProductSet| defines the Cartesian product of two
        or more sets. The product set is useful when representing higher
        dimensional spaces. For example, to represent a three-dimensional space,
        SymPy uses the Cartesian product of three real sets.

      \item \verb|ImageSet| represents the image of a function when applied to
        a particular set. The image set of a function $F$ with respect to a set
        $S$ is $\{ F(x) \mid x \in S \}$. SymPy uses image sets to represent sets
        of infinite solutions of equations such as $\sin(x)=0$.


      \item \verb|ConditionSet| represents a subset of a set whose members
        satisfy a particular condition. The subset of set $S$
        given by the condition $H$ is $\{x \mid H(x), x \in S \}$. SymPy uses
        condition sets to represent the set of solutions of equations and
        inequalities, where the equation or the inequality is the condition and
        the set is the domain over which it is being solved.

\end{itemize}

A few other classes are implemented as special cases of the classes described
above. The set of real numbers, \verb|Reals|, is implemented as a special case
of \verb|Interval|. \verb|ComplexRegion|
is implemented as a special case of \verb|ImageSet|. \verb|ComplexRegion|
supports both polar and rectangular representation of regions on the complex
plane.


\subsection{Category Theory}
SymPy includes a basic version of the module for dealing with categories ---
abstract mathematical objects representing classes of structures as classes of
objects (points) and morphisms (arrows) between the objects.  This version of
the module was designed with the following two goals in mind:

\begin{enumerate}
\item automatic typesetting of diagrams given by a collection of
  objects and of morphisms between them, and
\item specification and (semi-)automatic derivation of properties
  using commutative diagrams.
\end{enumerate}

At of version 1.0, SymPy only implements the first goal, while a (very partially
working) draft of implementation of the second goal is available
at~\cite{ct4commutativity}.

In order to achieve the two goals, the module \texttt{categories} defines
several classes representing some of the essential concepts: objects, morphisms,
categories, and diagrams.  In category theory, the inner structure of objects is
often discarded in the favor of studying the properties of morphisms, so the
class \texttt{Object} is essentially a synonym of the class \texttt{Symbol}.
There are several morphism classes which do not have a particular internal
structure either, though an exception is \texttt{CompositeMorphism}, which
essentially stores a list of morphisms.

To capture the properties of morphisms, the class \texttt{Diagram} is expected
to be used.  This class stores a family of morphisms, the corresponding source
and target objects, and, possibly, some properties of the morphisms.  Generally,
no restrictions are imposed on what the properties may be --- for example, one
might use strings of the form ``forall'', ``exists'', ``unique'', etc.
Furthermore, the morphisms of a diagram are grouped into \textit{premises} and
\textit{conclusions}, in order to be able to represent logical implications of
the form ``for a collection of morphisms $P$ with properties $p:P\to \Omega$ (the
premises), there exists a collection of morphisms $C$ with properties $c:C\to
\Omega$ (the conclusions),'' where $\Omega$ is the universal collection of
properties.  Finally, the class \texttt{Category} includes a collection of
diagrams which are deemed commutative and which therefore define the properties
of this category.

Automatic typesetting of diagrams takes a \texttt{Diagram} and produces \LaTeX{}
code using the \texttt{Xy-pic} package.  Typesetting is done in two stages:
layout and generation of \texttt{Xy-pic} code.  The layout stage is taken care
of by the class \texttt{DiagramGrid}, which takes a \texttt{Diagram} and lays out
the objects in a grid, trying to reduce the average length of the arrows in the
final picture.  By default, \texttt{DiagramGrid} uses a series of triangle-based
heuristics to produce a rectangular grid.  A linear layout can also be imposed.
Furthermore, groups of objects can be given; in this case, the groups will be
treated as atomic cells, and the member objects will be typeset independently of
the other objects.

The second phase of diagram typesetting consists of actually drawing the picture
and is carried out by the class \texttt{XypicDiagramDrawer}.  An example of a
diagram automatically typeset by \texttt{DiagramgGrid} and
\texttt{XypicDiagramDrawer} in given in Figure~\ref{fig:cat:loops}.
\begin{figure}[h]
  \centerline{
    \xymatrix{
      A \ar[r]_{f} \ar@/^3mm/[rr]^{h_{2}} \ar@(u,l)[]^{l_{A}} \ar@/^3mm/@(l,d)[]^{n_{A}} & B \ar[d]^{g} & D \ar[l]^{k} \ar@/_7mm/[ll]_{h} \ar@/_11mm/[ll]_{h_{1}} \ar@(r,u)[]^{l_{D}} \ar@/^3mm/@(d,r)[]^{n_{D}} \\
      & C \ar@(l,d)[]^{l_{C}} \ar@/^3mm/@(d,r)[]^{n_{C}} &
    }
  }
  \caption{An automatically typeset commutative diagram}
  \label{fig:cat:loops}
\end{figure}

As far as the second main goal of the module is concerned, a (non-working) draft
of an implementation is at~\cite{ct4commutativity}.  The principal idea consists
of automatically deciding whether a diagram is commutative or not, given a
collection of ``axioms'' --- diagrams \textit{known} to be commutative.  The
implementation is based on graph embeddings (injective maps): whenever an
embedding of a commutative diagram into a given diagram is found, one concludes
that the subdiagram is commutative.  Deciding commutativity of the whole diagram
is therefore based (theoretically) on finding a ``cover'' of the target diagram
by embeddings of the axioms.  The na\"{i}ve implementation proved to be
prohibitively slow; a better optimized version is therefore in order, as well as
application of heuristics.

Contributions to automatic inference of commutativity of diagrams are welcome.
The source code (both the one in master and in \texttt{ct4-commutativity}) is
extensively documented.  Even more extensive explanations (including some
literary chatter) are given at~\cite{scolobb}.


\subsection{SymPy Gamma}\label{sympy-gamma}


SymPy Gamma is a simple web application that runs on Google App Engine.
It executes and displays the results of SymPy expressions as well as
additional related computations, in a fashion similar to that of
Wolfram\textbar{}Alpha. For instance, entering an integer will display
its prime factors, digits in the base-10 expansion, and a factorization
diagram. Entering a function will display its docstring; in general,
entering an arbitrary expression will display its derivative, integral,
series expansion, plot, and roots.

SymPy Gamma also has several features beyond just computing the
results using SymPy.

\begin{itemize}
\item
  SymPy Gamma displays integration and differentiation steps in detail, which
  can be viewed in Figure~\ref{fig:integralsteps}:\par
  {
    \centering
    \includegraphics[width=0.7\textwidth]{integral_steps.png}
    \captionof{figure}{Integral steps of $\tan (x)$}\label{fig:integralsteps}
    \par
  }
\item
  SymPy Gamma displays the factor tree diagrams for different numbers.
\item
  SymPy Gamma saves user search queries, and offers many such similar features
  for free, which Wolfram\textbar{}Alpha only offers to its paid users.
\end{itemize}
Every input query from the user on SymPy Gamma is first parsed by its
own parser, which handles several different forms of function names,
which SymPy as a library does not support. For instance, SymPy Gamma
supports queries like \texttt{sin\ x}, whereas SymPy does not support
this, and supports only \verb|sin(x)|.

This parser converts the input query to the equivalent SymPy readable code,
which is then eventually processed by SymPy, and the result is finally printed
with the built-in LaTeX output and rendered on the SymPy Gamma web-application.


\subsection{SymPy Live}\label{sympy-live}

SymPy Live is an online Python shell, which runs on the Google
App Engine that executes SymPy code. It is integrated in the SymPy
documentation examples at \href{http://docs.sympy.org}{http://docs.sympy.org}.

This is accomplished by providing a HTML/JavaScript GUI for entering
source code and visualization of output, and a server that
evaluates the requested source code. It is an interactive AJAX shell
that runs SymPy code using Python on the server.


\subsection{Comparison with Mathematica}


Wolfram Mathematica is a popular proprietary CAS\@.
It features highly advanced algorithms.
Mathematica has a core implemented in C++~\cite{Wolfram2016}
which interprets its own programming language, Wolfram Language.

% M-expressions

Analogous to Lisp S-expressions,
Mathematica uses its own style of M-expres\-sions,
which are arrays of either atoms or other M-expressions.
The first element of the expression identifies the type of the expression
and is indexed by zero, and the first argument is indexed starting with one.
In SymPy, expression arguments are stored in a Python tuple
(that is, an immutable array),
while the expression type is identified by the type of the object storing the
expression.

% Attributes

Mathematica can associate attributes to its atoms.
Attributes may define mathematical properties and behavior of the nodes
associated to the atom.
In SymPy, the usage of static class fields is roughly similar to Mathematica's
attributes, though other programming patterns may also be used the achieve an
equivalent behavior, such as class inheritance.

% Expression mutability

Unlike SymPy, Mathematica expressions are mutable,
that is, one can change parts of the expression tree without
creating a new object.
The mutability of Mathematica expressions allows for a lazy updating of any references
to a given data structure.

% * comparison with Mathematica: commutativity, associative expressions, one-identity. Advantage of SymPy: multiplicative commutativity defined on symbols.
% Products and commutativity

Products in Mathematica are determined by some built in node types, such as
\texttt{Times}, \texttt{Dot}, and others.  \texttt{Times} is a representation of
the \texttt{*} operator, and is always meant to represent a commutative product
operator.  The other notable product is \texttt{Dot}, which represents the
\texttt{.} operator.  This product represents matrix multiplication. It is not
commutative.  Unlike
Mathematica, SymPy determines commutativity with respect to multiplication from
the expression type of the factors.  Mathematica puts the \texttt{Orderless} attribute
on the expression type.

% Associative expressions.

Regarding associative expressions,
SymPy handles associativity of sums and products by automatically flattening them,
Mathematica specifies the \texttt{Flat}~\cite{WolframRefFlat} attribute on the expression type.

% One identity


% Pattern matching

Mathematica relies heavily on pattern matching---even the so-called equivalent of function declaration is in reality
the definition of a pattern generating an expression tree transformation
on input expressions.
%
Mathematica's pattern matching is sensitive to
associative~\cite{WolframRefFlat}, commutative~\cite{WolframRefOrderless}, and
one-identity~\cite{WolframRefOneIdentity} properties of its expression tree
nodes~\cite{WolframRefFlatAndOrderlessFunctions}.
%
SymPy has various ways to perform pattern matching.
All of them play a lesser role in the CAS than in Mathematica
and are basically available as a tool to rewrite expressions.
The differential equation solver in SymPy somewhat relies on pattern matching to
identify the kind of differential equation, but it is envisaged to replace
that strategy with analysis of Lie symmetries in the future.
Mathematica's real advantage is the ability to add new overloading to the
expression builder at runtime, or for specific subnodes.
Consider for example:
\begin{verbatim}
In[1]:= Unprotect[Plus]
Out[1]= {Plus}

In[2]:= Sin[x_]^2 + Cos[y_]^2 := 1

In[3]:= x + Sin[t]^2 + y + Cos[t]^2
Out[3]= 1 + x + y
\end{verbatim}
This expression in Mathematica defines a substitution rule that overloads
the functionality of the \texttt{Plus} node (the node for additions in Mathematica).
The trailing underscore after a symbol means that it is to be considered a
wildcard.
This example may not be practical, as one may wish to keep this identity
unevaluated.  Nevertheless, it clearly illustrates the potential to define
immediate transformation rules.
In SymPy, the operations constructing the addition node in the expression tree
are Python class constructors
and cannot be modified at runtime.\footnote{In reality, Python supports monkey patching,
nonetheless, it is a discouraged programming pattern.}
The way SymPy deals with extending the missing runtime overloadability functionality
is by subclassing the node types.
Subclasses may redefine the class constructor to yield the proper
extended functionality.


%% TODO list:
% * comparison with Mathematica: MatrixExp, product not always commutative, type inheritance (polymorphism) and advantage in unifying the product symbol * for symbols and matrices, pattern matching vs. single dispatch.

% Type inheritance and polymorphism

Unlike SymPy, Mathematica does not support type inheritance or poly\-morph\-ism~\cite{Fateman1992}.
% cite examples of class inheritance in SymPy:
%
SymPy relies heavily on class inheritance, but for the most part,
class inheritance is used to make sure that SymPy objects inherit the proper
methods and implement the basic hashing system.
%There are also cases where inheritance is used to extend the mathematical meaning of an expression.

% Matrices

Matrices in SymPy are separate types from lists.
In Mathematica, nested lists are interpreted as matrices whenever the sublists
have the same length.
The main difference to SymPy is that ordinary operators and functions
do not get generalized the same way as used in traditional mathematics.
Using the standard multiplication in Mathematica performs an element-wise
product. This is compatible with Mathematica's convention of commutativity of
\texttt{Times} nodes.
Matrix product is expressed by the \textit{dot} operator,
or the \texttt{Dot} node.
The same is true for the other operators, and even functions,
most notably calling the exponential function \texttt{Exp} on a matrix
returns an element-wise exponentiation of its elements.
The real matrix exponential is available through the \texttt{MatrixExp}
function.

% * comparison with Mathematica: avoid misspelling variables through forced declaration (check that you can't do it in Mathematica).
% * evaluate=False vs HoldForm

Unevaluated expressions in Mathematica can be achieved in various ways,
most commonly with the \texttt{HoldForm} or \texttt{Hold} nodes,
that block the evaluation of subnodes by the parser.
Note that such a node cannot be expressed in Python, because of greedy evaluation.
Whenever needed in SymPy, it is necessary to add the parameter \texttt{evaluate=False}
to all subnodes, or put the input expression in a string.

% * comparison with Mathematica: == is structural equality, not

In Mathematica, the operator \texttt{==} returns a boolean whenever it is able
to immediately evaluate the truth of the equality, otherwise it returns an
\texttt{Equal} expression.  In SymPy, \texttt{==} means structural equality and
is always guaranteed to return a boolean expression.  To express an equality in
SymPy it is necessary to explicitly construct an object of the \texttt{Equality}
class.

% * comparison with Mathematica: polynomial module.
% * comparison with Mathematica: space is product, ** vs ^

SymPy, in accordance with Python and unlike the usual programming convention,
uses \texttt{**} to express the power operator, while Mathematica uses the more
common \verb|^|.

% * comparison with Mathematica: ( ) is Sequence, functions are generally uppercase.
% * comparsion with Mathematica: table of comparison?
% * comparison with Mathematica: Wolfram language has loads of operator overloading, functional paradigm.

SymPy's use of floating-point numbers is similar to that of most
other CASs, including Maple and Maxima.
By contrast, Mathematica uses a form
of significance arithmetic~\cite{Sofroniou2005precise} for approximate numbers.
This offers further protection against numerical errors,
although it comes with its own set of problems
(for a critique of significance arithmetic, see Fateman~\cite{Fateman1992}).
Internally, SymPy's \texttt{evalf} method works similarly to Mathematica's
significance arithmetic, but the semantics are isolated from the rest of the system.


\subsection{Other Projects that use SymPy}

\input{other_projects_that_use_sympy}


\subsection{Tensors}

Ongoing work to provide the capabilities of tensor computer algebra has so far
produced the \texttt{sympy.tensor} module.  It is composed of three separated
submodules, whose purposes are quite different: \texttt{tensor.indexed} and
\texttt{tensor.\allowbreak{}indexed\_methods} support indexed symbols,
\texttt{tensor.array} contains facilities to operator on symbolic $N$-dimensional
arrays, and finally \texttt{tensor.tensor} is used to define abstract tensors.
The abstract tensors subsection
is inspired by xAct~\cite{xAct} and Cadabra~\cite{Peeters2007cadabra}.
Canonicalization based on the Butler-Portugal~\cite{ManssurPortugal1999}
algorithm is supported in SymPy.  It is currently limited to polynomial tensor
expressions.



\subsection{Numerical simplification}

The \texttt{nsimplify} function in SymPy
(a wrapper of \texttt{identify} in mpmath)
attempts to find a simple symbolic
expression that evaluates to the same numerical value as the given
input.
It works by applying a few simple transformations
(including square roots, reciprocals, logarithms and exponentials) to
the input and, for each transformed value,
using the PSLQ algorithm~\cite{Ferguson1999} to search for
a matching algebraic number or optionally a linear combination
of user-provided base constants (such as $\pi$).

\begin{verbatim}
>>> t = 1 / (sin(pi/5)+sin(2*pi/5)+sin(3*pi/5)+sin(4*pi/5))**2
>>> nsimplify(t)
-2*sqrt(5)/5 + 1
>>> nsimplify(pi, tolerance=0.01)
22/7
>>> nsimplify(1.783919626661888, [pi], tolerance=1e-12)
pi/(-1/3 + 2*pi/3)
\end{verbatim}
